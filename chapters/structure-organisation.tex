\section{STRUCTURE AND ORGANISATION}\label{sec:structure-and-organisation}

\item \textbf{Bodies} \\
DAA's bodies are the:
\begin{itemize}
    \item DAA Ballot Vote ("Urabstimmung");
    \item DAA Chairman ("Vorstand");
    \item DAA Member Community;
\end{itemize}

\item \textbf{General Concept of Competences and Duties} \\
The objective of DAA is to establish a decentralised and democratic association with flat hierarchies.
Therefore, the DAA Chairman ("Vorstand") will have only those competences, which actually require the action of an individual, natural person, such as representation duties or the duty to keep the books.
The DAA Ballot Vote ("Urabstimmung") has those elementary competences, which are mandatory stipulated for an association assembly by Swiss law (such as the change of statutes, the liquidation of the association, and others).
The DAA Ballot Vote will be held digital, allowing the proper democratic decision-making required by Swiss law.
Thirdly, the DAA Member Community ("Mitglied") shall be the body, which can decide about the "daily business" such as the proposal and support of projects and the funding allocation to projects.
Every member of the association is also part of the DAA Member Community, which provides a blockchain-based technical infrastructure to efficiently, democratically and transparently propose and vote.

\item \textbf{Underlying Technology} \\
The DAA is technically built with Ethereum Smart Contracts.
All voting will take place on this technical infrastructure.
The relevant technical functions are hereinafter written in italic.
To be able to vote, holding a certain amount of Ether is necessary for every member (transaction fees, gas fees).

\item \textbf{DAA Chairman (Vorstand)} \\
The DAA has at minimum two (2) Chairmans with the following competencies and duties:
\begin{enumerate}
    [label=(\alph*)]
    \item Representing the DAA to the outside world;
    \item Keeping the books and creating the necessary financial statements of the DAA;
    \item Preparing and calling the next DAA Ballot Vote (\emph{setVotingSlot});
    \item Reviewing member applicants on their eligibility to join the DAA.
          If the member meets the requirements of Art. 11, the DAA Chairman will add their public keys to the member registry (*whitelistMember*);
    \item Remove Members who haven't paid their membership fees;
    \item Appointing the informal DAA Board of Advisors; %TODO
    \item Keeping the member registry (name, address, e-mail); %TODO
\end{enumerate}
The term of office for the DAA Chairmans starts with the foundation of this DAA and ends when he's been replaced via a proposal.
If a DAA Chairman becomes unable to act or loses his private key, he must be replaced with a new Chairman in the next DAA Ballot Vote.
If all the DAA chairmans are unable to act, an extraordinary DAA Ballot Vote will be called and two new DAA Chairmans must be elected.
The personal liability of a DAA Chairman is limited to cases of gross negligence.

\item \textbf{DAA Ballot Vote ("Urabstimmung")} \\
\begin{enumerate}[label=\textbf{\arabic*.}]
    \item \textbf{Competences} \\
    The DAA Ballot Vote shall be the highest governing body of the DAA.
    It is chaired by one of the DAA Chairmans.
    The DAA Ballot Vote has the duty of collecting all the votes on the proposals on that vote.
    The DAA Member Community has a one day timeslot to vote for every proposal in the Ballot Vote.

    \item \textbf{Voting Majorities} \\
    Resolutions shall be adopted by a simple majority of the members participating at the individual DAA Ballot Vote.
    At least 1/5 of the DAA Member Community must vote on a proposal to become successful.
    The update of the underlying code and the dissolution of the DAA require a majority of 2/3 of the DAA Member Community.

    \item \textbf{Election Process for DAA Chairman} \\
    Any member can propose himself for candidacy as chairman.
    If the proposal is successful the new DAA Chairman term starts at the execution of the proposal.

    \item \textbf{Convocation} \\
    The DAA Ballot Vote shall be held in regular intervals.
    A DAA Chairman can set the date at least 1 month before the DAA Ballot Vote (\emph{setVotingSlot}).
    The DAA Chairman will inform the members electronically on the place and date.
    A DAA Ballot Vote lasts one full day. \\ \\

    Every member can propose an extraordinary DAA Ballot Vote (\emph{propose}). %TODO
    The time slot for the vote (voteForGeneralAssemblyDate) is two weeks.
    If 20\% of all DAA members support the extraordinary DAA Ballot Vote, it will take place on the specified date.
    If the DAA Chairman is voting on behalf of the extraordinary DAA Assembly, it will take place irrespective of any majority or quorum. \\ \\
    The DAA Ballot Vote is held purely digital.

    \item \textbf{Proposals \& Agenda Items} \\
    Proposals from members must be submitted at least 7 days prior to the DAA Ballot Vote using the specific DAO proposal function (\emph{propose}).
\end{enumerate}


\item \textbf{DAA Member Community} \\
\begin{enumerate}[label=\textbf{\arabic*.}]
    \item \textbf{Members \& Purpose} \\
    Every member of the DAA is also member of the DAA Member Community.
    The objective of the DAA Member Community is to allow every member to permanently propose new DAA projects and to vote on the funding allocation to those projects. %TODO in doku schauen
    In this area, the DAA Member Community is the executive body of the DAA.
    The funding allocation via the DAA Member Community voting mechanism is binding and technically non-reversible – not even by the DAA Chairman or the DAA Ballot Vote.

    \item \textbf{Competences} \\
    The DAA Member Community has the following competences:
    \begin{enumerate}
        [label=(\alph*)]
        \item Proposing new projects and funding allocations (\emph{propose});
        \item Voting on those projects and the use of funds (\emph{castVote} or \emph{castVoteWithReason});
        \item Expelling DAA members (see Art. 17).
    \end{enumerate}

    \item \textbf{Proposal \& Voting Process} \\
    The whole voting process on the DAA Member Community is purely digital and block- chain-based.
    Every member can submit a proposal to pay Ether to a destination address (the amount can be zero).
    The votes require simple majority and a 1/5 quorum of the DAA Member Community.
    The time slot for a proposal vote is one day during the DAA Ballot Vote.
\end{enumerate}
