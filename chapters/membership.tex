\section{MEMBERSHIP}\label{sec:membership}

\item \textbf{Members \& Membership Requirements} \\
Natural persons, legal entities and organizations under public law can request membership of the DAA.
Legal entities and organizations under public law shall appoint a representative who exercises membership rights at the DAA Ballot Vote.
Every member is responsible to gain the technological know-how to be able to participate on votes on the DAA Assembly and the DAA Member Community.
In addition, every member has to assure to have a sufficient amount of Ether for the necessary transaction fees.

\item \textbf{Becoming a Member} \\
Everyone who is eligible for a membership can make a request(\emph{requestMembership}).
Every member has to do a KYC check.
A new member has to be whitelisted by at least two DAA Chairmans before joining(\emph{whitelistMember}).
After the whitelisting and the payment of the membership fee (\emph{payMembershipFee}), the applicant becomes a DAA member and gains voting power for the next ballot vote.

\item \textbf{Awareness of Technological \& Conceptual Risks} \\
Blockchain is a new technology.
The technical or conceptual structure of this DAA and the DAA voting process may have weaknesses, as it is the case with every blockchain project.
Moreover, the DAA is dependent on the underlying Ethereum protocol.
Therefore, it may be possible that the DAA loses part or the whole of its funds or become incapable of acting.
Every member explicitly declares to be aware of and to agree to those risks.
